\section{Apéndices}

	\subsection{Apéndice A: Enunciado}
		%\input{../Enunciado_TP1.pdf}
	%\clearpage

	\subsection{Apéndice B: Código fuente numericamente relevante}
	\clearpage

	\subsection{Apéndice C: Demostración $A = pWD + ez^{t}$}

		Se quiere ver que $A = pWD + e z^{t}$ \\

		Vamos a demostrar la igualdad para $1 \leq i \leq n$ y $1 \leq j \leq n$. \\

		\[ \text{Sea $A \in \mathbb{R}^{nxn} /$ } a_{ij} =
		        \begin{cases}
		                (1-p)/n + (p \, w_{ij})         & \text{si } c_{j}  \neq 0 \\
		                1    /n                         & \text{si } c_{j}   =   0
		        \end{cases}
		\]

		\[ \text{y sea $D \in \mathbb{R}^{nxn}$ una matriz diagonal $/$ } d_{ij} =
		        \begin{cases}
		                1/c_j         & \text{si } c_{j}  \neq 0 \\
		                1/n           & \text{si } c_{j}   =   0
		        \end{cases}
		\]

		Como multiplicar por derecha por una matriz diagonal (en este caso $D$), es multiplicar a cada columna de $W$ por elementos de la diagonal de $D$, entonces: \\

		\[ (pWD)_{ij} =
		        \begin{cases}
		                (p \, w_{ij})/c_j 	& \text{si } c_{j}  \neq 0 \\
		                0 			& \text{si } c_{j}   =   0
		        \end{cases}
		\]

		Notar que $p$ es un escalar, y la operación $WD$ se puede hacer porque ambas matrices pertenecen a $\mathbb{R}^{nxn}$ (ver las definiciones de las matrices en la introducción teórica o en el enunciado del trabajo práctico). \\

		\[ \text{Sean }
		        e       = \begin{pmatrix}
		                        1 \\
		                        \vdots \\
		                        1
		                \end  {pmatrix}
		\qquad
			\text{ y }
		\qquad
			z_{j} = \begin{cases}
		                        (1-p)/n & \text{si } c_{j} \neq 0 \\
		                         1   /n & \text{si } c_{j}   =  0
				\end  {cases}
		\]

		Entonces $e z^{t}$ es una matriz en $\mathbb{R}^{nxn}$, con $z^{t}$ en cada fila: \\

		\[
		        e z^{t} 	= 	\begin{pmatrix}
							1 \\
							\vdots \\
							1
						\end  {pmatrix}
						\begin{matrix}
							\begin{pmatrix}z_1 & \hdots & z_n
							\end  {pmatrix}\\\mbox{}
						\end{matrix}
					= 	\begin{pmatrix}
							z^t 	\\
							z^t 	\\
							\vdots 	\\
							z^t
						\end  {pmatrix}
		\]

		\[
			        ez^{t} = \begin{cases}
						(1-p)/n & \text{si } c_{j} \neq 0 \\
						 1   /n & \text{si } c_{j}   =  0
					 \end  {cases}
		\]

		Entonces $pWD \in \mathbb{R}^{n}$ y $ez^{t} \in \mathbb{R}^{nxn} =>$ se pueden sumar. \\

		Los valores de ambas dependen de $c_j => pWD+ez^t$ queda definida como:

		\[
			        pWD+ez^{t} = 	\begin{cases}
							pw_{ij}/c_j+(1-p)/n & \text{si } c_{j} \neq 0 \\
								     1   /n & \text{si } c_{j}   =  0
						\end  {cases}
		\]

		Por conmutatividad de la suma en el caso $c_j \neq 0$ podemos ver que se trata efectivamente de la matriz $A$. \\

		\qed

	\clearpage

	\subsection{Apéndice D: Aplicabilidad E.G., condicionamiento de la matriz $(I-pWD)$ e influencia del valor $p$ en ello}

	\clearpage

