\section{Desarrollo}

% Deben explicarse los métodos numéricos que utilizaron y su aplicación al problema
% concreto involucrado en el trabajo práctico. Se deben mencionar los pasos que si-
% guieron para implementar los algoritmos, las dificultades que fueron encontrando y la
% descripción de cómo las fueron resolviendo. Explicar también cómo fueron planteadas
% y realizadas las mediciones experimentales. Los ensayos fallidos, hipótesis y conjeturas
% equivocadas, experimentos y métodos malogrados deben figurar en esta sección, con
% una breve explicación de los motivos de estas fallas (en caso de ser conocidas).

	\subsection{Elección de la estructura de datos}

		Analizando las operaciones necesarias para la resolución del problema, llegamos a la conclusión de que para el armado del sistema a resolver convenía un tipo de estructura y para su efectiva resolución, otra. \\

		Concretamente, notamos que la matriz de conectividad $W$ era multiplicada por derecha por una matriz diagonal y luego se le sumaba una matriz identidad $I$: \\

		$(I - pWD) x = [W (-pD) + I] x = e$ \\

		Multiplicar a $W$ por una matriz diagonal $-pD$, es equivalente a multiplicar a cada columna $\vec{w}_i$ de $W$ por el escalar que está en la diagonal de $-pD$. \\

		$ W (-pD) = \left( \vec{w}_1 \vec{w}_2 \hdots \vec{w}_n \right) (-pD) = \left( -p d_{11} \vec{w}_1  -p d_{22} \vec{w}_2  \hdots -p d_{nn} \vec{w}_n \right)$ \\

		Viendo esto, pensamos que sería bueno tener alguna estructura que contuviera a \textit{las columnas} de $W$, para poder pedirlas en $O(1)$ y multiplicar fácilmente a cada uno de sus elementos por un mismo escalar. \\

		En particular, vamos a querer multiplicar a sus elementos \textit{distintos de $0$} (pues los que son $0$ no haría falta porque quedarían iguales), y la matriz $W$, por ser rala, contiene muchos elementos iguales a $0$. Es por esto que decidimos usar un \textit{map<int, double>} para almacenar las columnas: donde el \textit{int} es el número de fila y el \textit{double} es el valor almacenado en esa fila del \textit{map} (es decir, de la columna). \\

		Por otro lado, para la triangulación del sistema lineal mediante el algoritmo de Eliminación Gaussiana, lo que sucede es más bien lo opuesto: vamos a premultiplicar (multiplicar por izquierda) repetidas veces al sistema $(I - pWD) x = e$ por \textit{matrices elementales}. \\

	\subsection{Page Rank}

	\subsection{Eliminación Gaussiana}

	\subsection{Substitución hacia atrás}

