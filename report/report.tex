%Clase y configuracion de tipo de documento
\documentclass[10pt,a4paper,spanish]{article}
% Inclusion de paquetes
\usepackage{a4wide}
\usepackage{amsmath, amscd, amssymb, amsthm, latexsym}
\usepackage[spanish]{babel}
\usepackage[utf8]{inputenc}
\usepackage[width=15.5cm, left=3cm, top=2.5cm, height= 24.5cm]{geometry}
\usepackage{fancyhdr}
\pagestyle{fancyplain}
\usepackage{listings}
\usepackage{enumerate}
\usepackage{xspace}
\usepackage{longtable}
\usepackage{caratula}
\usepackage[utf8]{inputenc}
\usepackage[spanish]{babel}
\usepackage{caption}
\usepackage{subcaption}
\usepackage{float}
\usepackage{url}
\usepackage{esvect} % para usar \vec{w_i} en development.tex

% Encabezado
\lhead{M\'etodos Num\'ericos}
%\rhead{Grupo GGJP}
% Pie de pagina
\renewcommand{\footrulewidth}{0.4pt}
\lfoot{Facultad de Ciencias Exactas y Naturales}
\rfoot{Universidad de Buenos Aires}

\begin{document}

% from pautas.pdf:
% La primera página del informe comenzará con el nombre de la universidad, la facultad, el
% departamento y la materia. A continuación, el título del trabajo, el nombre y direcciones
% de correo electrónico de los autores, el resumen y las palabras clave.
% El título debera ser breve y apropiado para una rápida identificación del contenido del
% trabajo. El resumen, de no mas de 200 palabras, debera explicar brevemente el trabajo
% realizado y las conclusiones de los autores de manera que pueda ser util por si solo para
% dar una idea del contenido del trabajo. Las palabras clave, no mas de cuatro, deben ser
% términos técnicos que den una idea del contenido del trabajo para facilitar su búsqueda
% en una base de datos temática.

% Datos de carátula

\materia{Métodos Numéricos}
\titulo{Trabajo Práctico 1}
\subtitulo{Sistemas de ecuaciones lineales}
%\grupo{Grupo: GGJP} 
\integrante{Giudice, Carlos}{694/15}{carlosr.giudice@gmail.com}
\integrante{Grenier, Michelle}{520/12}{michelle.grenier@hotmail.com}
\integrante{Junqueras, Juan}{804/16}{juanjunqueras@gmail.com}
\integrante{Pyrih, Franco}{520/12}{fpyrih@dc.uba.ar}

\maketitle

\section{Resumen} % no más de 200 palabras % son 83

Este trabajo se centra en el estudio del algoritmo PageRank, propuesto en 1998 e implementado por el reconocido motor de búsqueda de Google, para clasificar sitios web por orden de relevancia. \\

Se encontraran los detalles de su implementación, incluyendo la elección de una estructura de datos apropiada para trabajar eficientemente con una gran cantidad de datos. \\

También se llevará a cabo un análisis de la \textit{performance} del algoritmo, teniendo en cuenta tanto el tiempo de cómputo, como la calidad de los resultados que produce.

\paragraph{Palabras clave:} % no más de 4 palabras
PageRank, sparse matrix, ranking, websites
% algorithms, big data, main eigenvector
\newpage

% Para crear un índice
\tableofcontents

% Forzar salto de página
\clearpage

\section{Introducción}

% from file pautas.pdf:

	% Contendrá una breve explicación de la base teórica que fundamenta los métodos involucrados
	% en el trabajo, junto con los métodos mismos. No deben incluirse demostraciones
	% de propiedades ni teoremas, ejemplos innecesarios, ni definiciones elementales (como
	% por ejemplo la de matriz simétrica). En vez de definiciones básicas es conveniente citar
	% ejemplos de bibliografía adecuada. Una cita vale más que mil palabras.

% from file Enunciado_TP1.pdf, Enunciado section:

	% Previamente, deberán estudiar las caracterı́sticas de la matriz involucrada y responder a lo siguiente:
	% 1. ¿Por qué la matriz A definida en (4) es equivalente a pWD+ez^t? Justificar adecuadamente.
	% 2. ¿Cómo se garantiza la aplicabilidad de EG? ¿La matriz I-pWD está bien condicionada? ¿Cómo influye el valor de p?

Un motor de búsqueda debe poder ordenar las páginas web de acceso público según su importancia, para poder presentar los resultados de las búsquedas de una manera más útil. Este trabajo se centra en cómo determinar ese orden. \\

Para esto vamos a usar la matriz que representa cómo enlazan las distintas páginas entre si y explotar sus propiedades numéricas. \\

No solo vamos a ponderar la cantidad de enlaces que llegan a una página, sino la relevancia de las páginas que enlazan a las mismas. \\

Para un conjunto de páginas web: $\{1 ... n\}$ vamos a definir a la \textbf{matriz de conectividad} $ W \in \{0, 1\}^{nxn} $ como: \\

\[ W_{ij} =
	\begin{cases}
		1 & \text{ si la página \textit{j} tiene un enlace a la página \textit{i} } \\
		0 & \text{ si la página \textit{j} \textbf{no} tiene un enlace a la página \textit{i} } \\
		0 & \text{ si $j = i$}
	\end{cases}
\]

Es decir, que en la fila \textit{i} están las páginas que apuntan a la página \textit{i} y en la columna \textit{j} están las páginas apuntadas por \textit{j}. \\

Luego, para una página $j \in W$, definimos su \textbf{grado} $c_{j} = \sum_{i = 1}^{n} w_{ij}$. Es decir, como la cantidad de enlaces \textit{salientes} de \textit{j}. \\

Para ordenar las páginas web por orden de importancia, vamos a asignarle a cada página \textit{i} un \textbf{puntaje} $x_i$. \\

Dadas $i, j \in \{1...n\}$ el \textbf{aporte} del enlace que va de la página $j$ a la página $i$ se calcula como: \\

\[
	\begin{cases}
		\frac{x_j}{c_j}w_{ij} 	& \text{ si la página \textit{j} tiene un enlace a la página \textit{i} } \\
		0 			& \text{ si la página \textit{j} \textbf{no} tiene un enlace a la página \textit{i} } \\
		0 			& \text{ si el grado $c_j = 0$}
	\end{cases}
\]

Es decir que, cuando existe un enlace desde \textit{j} hacia \textit{i} y el grado de \textit{j} $c_j$ es distinto de $0$, la página \textit{j} le aporta a la página \textit{i} su puntaje, ponderado por cuántos enlaces salientes tiene. \\

Finalmente, calculamos el puntaje de la página \textit{i} como:

$ x_i = \sum_{j=1}^{n} \frac{x_j}{c_j} w_{ij} $

Notar que el puntaje de una página depende del puntaje de las otras. \\

Para modelar mejor el comportamiento de los usuarios de la web, se introduce el modelo del \textit{navegante aleatorio}: la idea es que el usuario incia en una página cualquiera y luego sigue uno de sus enlaces con probabilidad $p$, o bien salta a una página al azar con probabilidad $1 - p$. Si la página actual no tiene enlaces, también salta a una página al azar (con probabilidad $1/n$). \\

Finalmente, definimos la matriz $A \in \mathbb{R}^{nxn}$, donde la posición $a_{ij}$ representa la probabilidad de pasar a la página \textit{i} estando en la página \textit{j} y está definida como:

\[ a_{ij} =
	\begin{cases}
		(1-p)/n + (p \, w_{ij}) 	& \text{si } c_{j}  \neq 0 \\
		1/n 				& \text{si } c_{j}   =   0
	\end{cases}
\]


\clearpage

\section{Desarrollo}

% Deben explicarse los métodos numéricos que utilizaron y su aplicación al problema
% concreto involucrado en el trabajo práctico. Se deben mencionar los pasos que si-
% guieron para implementar los algoritmos, las dificultades que fueron encontrando y la
% descripción de cómo las fueron resolviendo. Explicar también cómo fueron planteadas
% y realizadas las mediciones experimentales. Los ensayos fallidos, hipótesis y conjeturas
% equivocadas, experimentos y métodos malogrados deben figurar en esta sección, con
% una breve explicación de los motivos de estas fallas (en caso de ser conocidas).

	\subsection{Elección de la estructura de datos}

		Analizando las operaciones necesarias para la resolución del problema, pensamos que para el armado del sistema lineal iba a convenir un tipo de estructura y para su efectiva resolución, otra. \\

		Concretamente, notamos que la matriz de conectividad $W$ era multiplicada (por derecha) por una matriz diagonal y luego se le sumaba una matriz identidad $I$: \\

		$(I - pWD) x = [W (-pD) + I] x = e$ \\

		Multiplicar a $W$ por una matriz diagonal $-pD$, es equivalente a multiplicar a cada columna $\vec{w}_i$ de $W$ por el escalar que está en la diagonal de $-pD$: \\

		$ W (-pD) = \left( \vec{w}_1 \vec{w}_2 \hdots \vec{w}_n \right) (-pD) = \left( -p d_{11} \vec{w}_1 \quad -p d_{22} \vec{w}_2 \quad \hdots \quad -p d_{nn} \vec{w}_n \right)$ \\

		Viendo esto, pensamos que sería bueno tener alguna estructura que contuviera a \textit{las columnas} de $W$, para poder pedirlas en $O(1)$ y multiplicar fácilmente a cada una por un mismo escalar. \\

		En particular, vamos a querer multiplicar a sus elementos \textit{distintos de $0$} (pues los que son $0$ no haría falta porque quedarían iguales), y la matriz $W$, por ser rala, contiene muchos elementos iguales a $0$. Es por esto que pensamos usar un \textit{map$<$int, double$>$} para almacenar las columnas: donde el \textit{int} es el número de fila y el \textit{double} es el valor almacenado en esa posición del \textit{map} (es decir, de la columna). Nos evitaremos almacenar los $0$s, asumiendo que las posiciones que no están almacenadas son $0$. \\

		Por otro lado, para la triangulación del sistema lineal mediante el algoritmo de Eliminación Gaussiana, lo que sucede es más bien lo opuesto: cada paso del algoritmo es equivalente a \textit{pre}multiplicar (multiplicar por izquierda) al sistema $(I - pWD) x = e$ por un matriz elemental $E_{(i,j)}$ distinta. Estas matrices son como la matriz identidad, pero poseen un elemento distinto de $0$ en la posición $(i,j)$. \\

		$ (I - pWD) x = e <=> E_{(n,n-1)} \hdots E_{(2,1)} (I - pWD) x = E_{(n,n-1)} \hdots E_{(2,1)} e $ \\

		Esto, a su vez, es equivalente a hacer operaciones con las filas. Por ejemplo, si $E_{(2,1)}$ tiene un escalar $m_{2,1} \neq 0$ en la posición $(2,1)$, premultiplicar $E_{(2,1)}$ por $(I - pWD)$ será modificar la segunda fila de $(I - pWD)$ de manera tal que: \\

		$fila_{2}(I-pWD) = m_{2,1} *  fila_{1}(I-pWD) + fila_{2}(I-pWD)$ \\

		Esto nos hizo pensar que, para hacer estas operaciones, sería más fácil tener a la matriz como \textit{map} \underline{de filas}. \\

		Entonces lo que resolvimos hacer fue cargar la matriz como \textit{map} de columnas para armar el sistema, y luego trasponerla e interpretarla como un \textit{map} de filas para la resolución del sistema lineal. \\

	\subsection{\textit{Page Rank} (ranking de Page/rango de página)}

		Como vimos en la introducción teórica, para implementar Page Rank vamos a armar la matriz dispersa W, llamada \textit{de conectividad}, que contiene un 1 en la posición $(i,j)$ si la página \textit{j} enlaza a la página \textit{i}. \\

		El programa recibirá por parámetro la ruta del archivo que contiene la información de las páginas: en la primera línea la cantidad de enlaces, en la segunda la cantidad de páginas, y en la tercera y las subsiguientes, dos números separados por un espacio, que representan que existe un enlace desde el primero hacia el segundo. \\

		La cantidad de enlaces será usada para verificación, la cantidad de páginas la usaremos para definir el tamaño de la matriz de conectividad y, a medida que vayamos leyendo los enlaces, escribiremos $1$ en la posición de $W$ correspondiente, al tiempo que sumaremos $1$ al contador ubicado en la fila $i$ del vector $c$ de grados. \\


		\subsubsection{Eliminación Gaussiana \textit{(gaussian elimination)}}



		\subsubsection{\textit{Backwards Substitution} (substitución hacia atrás)}




\clearpage

\include{sections/experimentation}
\clearpage

\include{sections/conclusion}
\clearpage

\section{Apéndices}

	\subsection{Apéndice A: Enunciado}
		%\input{../Enunciado_TP1.pdf}
	%\clearpage

	\subsection{Apéndice B: Código fuente numericamente relevante}
	\clearpage

	\subsection{Apéndice C: Demostración $A = pWD + ez^{t}$}

		Se quiere ver que $A = pWD + e z^{t}$ \\

		Vamos a demostrar la igualdad para $1 \leq i \leq n$ y $1 \leq j \leq n$. \\

		\[ \text{Sea $A \in \mathbb{R}^{nxn} /$ } a_{ij} =
		        \begin{cases}
		                (1-p)/n + (p \, w_{ij})         & \text{si } c_{j}  \neq 0 \\
		                1    /n                         & \text{si } c_{j}   =   0
		        \end{cases}
		\]

		\[ \text{y sea $D \in \mathbb{R}^{nxn}$ una matriz diagonal $/$ } d_{ij} =
		        \begin{cases}
		                1/c_j         & \text{si } c_{j}  \neq 0 \\
		                1/n           & \text{si } c_{j}   =   0
		        \end{cases}
		\]

		Como multiplicar por derecha por una matriz diagonal (en este caso $D$), es multiplicar a cada columna de $W$ por elementos de la diagonal de $D$, entonces: \\

		\[ (pWD)_{ij} =
		        \begin{cases}
		                (p \, w_{ij})/c_j 	& \text{si } c_{j}  \neq 0 \\
		                0 			& \text{si } c_{j}   =   0
		        \end{cases}
		\]

		Notar que $p$ es un escalar, y la operación $WD$ se puede hacer porque ambas matrices pertenecen a $\mathbb{R}^{nxn}$ (ver las definiciones de las matrices en la introducción teórica o en el enunciado del trabajo práctico). \\

		\[ \text{Sean }
		        e       = \begin{pmatrix}
		                        1 \\
		                        \vdots \\
		                        1
		                \end  {pmatrix}
		\qquad
			\text{ y }
		\qquad
			z_{j} = \begin{cases}
		                        (1-p)/n & \text{si } c_{j} \neq 0 \\
		                         1   /n & \text{si } c_{j}   =  0
				\end  {cases}
		\]

		Entonces $e z^{t}$ es una matriz en $\mathbb{R}^{nxn}$, con $z^{t}$ en cada fila: \\

		\[
		        e z^{t} 	= 	\begin{pmatrix}
							1 \\
							\vdots \\
							1
						\end  {pmatrix}
						\begin{matrix}
							\begin{pmatrix}z_1 & \hdots & z_n
							\end  {pmatrix}\\\mbox{}
						\end{matrix}
					= 	\begin{pmatrix}
							z^t 	\\
							z^t 	\\
							\vdots 	\\
							z^t
						\end  {pmatrix}
		\]

		\[
			        ez^{t} = \begin{cases}
						(1-p)/n & \text{si } c_{j} \neq 0 \\
						 1   /n & \text{si } c_{j}   =  0
					 \end  {cases}
		\]

		Entonces $pWD \in \mathbb{R}^{n}$ y $ez^{t} \in \mathbb{R}^{nxn} =>$ se pueden sumar. \\

		Los valores de ambas dependen de $c_j => pWD+ez^t$ queda definida como:

		\[
			        pWD+ez^{t} = 	\begin{cases}
							pw_{ij}/c_j+(1-p)/n & \text{si } c_{j} \neq 0 \\
								     1   /n & \text{si } c_{j}   =  0
						\end  {cases}
		\]

		Por conmutatividad de la suma en el caso $c_j \neq 0$ podemos ver que se trata efectivamente de la matriz $A$. \\

		\qed

	\clearpage

	\subsection{Apéndice D: Aplicabilidad E.G., condicionamiento de la matriz $(I-pWD)$ e influencia del valor $p$ en ello}

	\clearpage



\end{document}
