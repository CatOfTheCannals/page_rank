%Clase y configuracion de tipo de documento
\documentclass[10pt,a4paper,spanish]{article}
% Inclusion de paquetes
\usepackage{a4wide}
\usepackage{amsmath, amscd, amssymb, amsthm, latexsym}
\usepackage[spanish]{babel}
\usepackage[utf8]{inputenc}
\usepackage[width=15.5cm, left=3cm, top=2.5cm, height= 24.5cm]{geometry}
\usepackage{fancyhdr}
\pagestyle{fancyplain}
\usepackage{listings}
\usepackage{enumerate}
\usepackage{xspace}
\usepackage{longtable}
\usepackage{caratula}
\usepackage[utf8]{inputenc}
\usepackage[spanish]{babel}
\usepackage{caption}
\usepackage{subcaption}
\usepackage{float}
\usepackage{url}

% Encabezado
\lhead{M\'etodos Num\'ericos}
%\rhead{Grupo GGJP}
% Pie de pagina
\renewcommand{\footrulewidth}{0.4pt}
\lfoot{Facultad de Ciencias Exactas y Naturales}
\rfoot{Universidad de Buenos Aires}

\begin{document}

% La primera pagina del informe comenzara con el nombre de la universidad, la facultad, el
% departamento y la materia. A continuacion, el titulo del trabajo, el nombre y direcciones
% de correo electronico de los autores, el resumen y las palabras clave.
% El titulo debera ser breve y apropiado para una rapida identificacion del contenido del
% trabajo. El resumen, de no mas de 200 palabras, debera explicar brevemente el trabajo
% realizado y las conclusiones de los autores de manera que pueda ser util por si solo para
% dar una idea del contenido del trabajo. Las palabras clave, no mas de cuatro, deben ser
% terminos tecnicos que den una idea del contenido del trabajo para facilitar su busqueda
% en una base de datos tematica.

% Datos de caratula

\materia{M\'etodos Num\'ericos}
\titulo{Trabajo Pr\'actico 1}
\subtitulo{Sistemas de ecuaciones lineales}
%\grupo{Grupo: GGJP} 
\integrante{Giudice, Carlos}{694/15}{carlosr.giudice@gmail.com}
\integrante{Grenier, Michelle}{520/12}{michelle.grenier@hotmail.com}
\integrante{Junqueras, Juan}{804/16}{juanjunqueras@gmail.com}
\integrante{Pyrih, Franco}{520/12}{fpyrih@dc.uba.ar}

\maketitle

\section{Resumen} % no más de 200 palabras % son 197 :D

\paragraph{Palabras clave:} % no más de 4 palabras
PageRank, sparse matrix, ranking, websites
% algorithms, big data, main eigenvector
\newpage

% Para crear un indice
\tableofcontents

% Forzar salto de pagina
\clearpage

\section{Introducción}


\clearpage

\section{Desarrollo}


\clearpage

\section{Experimentación}

	\subsection{Tiempo de cómputo}

	\subsection{Convergencia}

	\subsection{Calidad de los resultados}


\clearpage

\section{Conclusiones}
%Esta seccion debe contener las conclusiones generales del trabajo. Se deben mencionar las relaciones de la discusion sobre las que se tiene certeza, junto con comentarios y observaciones generales aplicables a todo el proceso. Mencionar tambi ́en posibles extensiones a los m ́etodos, experimentos que hayan quedado pendientes, etc.


\clearpage

\section{Apéndices}
%\subsection{Apéndice A: Enunciado}
%\input{enunciado}



\end{document}
